% cumtbthesis.tex
% CUMTB LaTeX 模板使用手册

\documentclass[master]{cumtbthesis}

% 使用手册不需要参考文献
% \addbibresource{ref/refs.bib}

% 手册信息
\ctitle{中国矿业大学(北京)毕业论文 \LaTeX{} 模板使用手册}
\cauthor{向睿超}
\studentid{}
\school{人工智能学院}
\major{人工智能}
\advisor{}
\advisortitle{}
\cdate{2025年12月}

\etitle{CUMTB Thesis \LaTeX{} Template User Manual}
\eauthor{Xiang RuiChao}
\eschool{School of Artificial Intelligence}
\emajor{Artificial Intelligence}
\eadvisor{}

\keywords{LaTeX;模板;使用手册;中国矿业大学}
\enkeywords{LaTeX; Template; User Manual; CUMTB}

\begin{document}

% 封面
\makecover

% 前置部分
\frontmatter

% 目录
\tableofcontents

% 主体部分
\mainmatter

\chapter{模板简介}

\section{关于本模板}

中国矿业大学(北京)毕业论文 \LaTeX{} 模板(cumtbthesis)是为帮助中国矿业大学(北京)的本科生、硕士生和博士生撰写学位论文而开发的 \LaTeX{} 文档类模板。

\subsection{设计目标}

\begin{itemize}
  \item \textbf{规范性}:严格遵循中国矿业大学(北京)的论文格式要求
  \item \textbf{易用性}:提供简洁的用户接口,降低使用门槛
  \item \textbf{专业性}:提供完整的学位论文撰写功能
  \item \textbf{可扩展性}:采用模块化设计,便于功能扩展
\end{itemize}

\subsection{主要特性}

\begin{enumerate}
  \item 支持本科、硕士、博士三种学位类型
  \item 盲审模式,一键生成盲审版本
  \item Makefile 和 latexmk 自动化构建
  \item 完整的符号说明、图表索引功能
  \item 原创性声明和版权授权书
  \item 使用 biblatex + biber 管理参考文献
\end{enumerate}

\section{版本历史}

\begin{table}[htbp]
  \centering
  \caption{版本更新历史}
  \begin{tabular}{lll}
    \toprule
    版本 & 日期 & 主要更新 \\
    \midrule
    v1.0 & 2025-12-05 & 初始版本,基础功能 \\
    v2.0 & 2025-12-05 & 新增6大核心功能 \\
    \bottomrule
  \end{tabular}
\end{table}

\chapter{安装与配置}

\section{系统要求}

\subsection{必需软件}

\begin{itemize}
  \item TeX Live 2020 或更新版本(推荐使用最新版)
  \item XeLaTeX 编译引擎
  \item Biber 参考文献处理器(可选)
\end{itemize}

\subsection{推荐软件}

\begin{itemize}
  \item \textbf{编辑器}:TeXstudio、VS Code + LaTeX Workshop、Overleaf
  \item \textbf{PDF 阅读器}:Skim(macOS)、SumatraPDF(Windows)
  \item \textbf{字体}:宋体、黑体、楷体、Times New Roman
\end{itemize}

\section{安装 TeX Live}

\subsection{macOS}

下载并安装 MacTeX(TeX Live 完整版 for macOS):

\begin{lstlisting}[language=bash]
# 使用 Homebrew
brew install --cask mactex

# 或从官网下载
# https://www.tug.org/mactex/
\end{lstlisting}

\subsection{Windows}

从官网下载 TeX Live 安装程序:

\begin{verbatim}
https://www.tug.org/texlive/acquire-netinstall.html
\end{verbatim}

\subsection{Linux}

\begin{lstlisting}[language=bash]
# Debian/Ubuntu
sudo apt-get install texlive-full

# Fedora
sudo dnf install texlive-scheme-full
\end{lstlisting}

\section{获取模板}

\subsection{从 GitHub 下载}

\begin{lstlisting}[language=bash]
git clone https://github.com/username/cumtb-thesis.git
cd cumtb-thesis
\end{lstlisting}

\subsection{下载 ZIP 压缩包}

访问 GitHub 项目页面,点击「Code」→「Download ZIP」。

\chapter{快速开始}

\section{配置论文信息}

编辑 \texttt{main.tex},修改以下内容:

\begin{lstlisting}[language=TeX]
% 选择学位类型
\documentclass[master]{cumtbthesis}
% bachelor - 本科
% master  - 硕士(默认)
% doctor  - 博士

% 中文信息
\ctitle{论文题目}
\cauthor{作者姓名}
\studentid{学号}
\school{学院名称}
\major{专业名称}
\advisor{导师姓名}
\advisortitle{导师职称}
\cdate{2025年6月}

% 英文信息
\etitle{Thesis Title}
\eauthor{Author Name}

% 关键词
\keywords{关键词1;关键词2;关键词3}
\enkeywords{keyword1; keyword2; keyword3}
\end{lstlisting}

\section{编译论文}

\subsection{使用 Makefile(推荐)}

\begin{lstlisting}[language=bash]
make thesis    # 编译论文
make clean     # 清理辅助文件
make help      # 查看所有命令
\end{lstlisting}

\subsection{使用 latexmk}

\begin{lstlisting}[language=bash]
latexmk -xelatex main.tex
\end{lstlisting}

\subsection{手动编译}

\begin{lstlisting}[language=bash]
xelatex main.tex
biber main
xelatex main.tex
xelatex main.tex
\end{lstlisting}

\section{第一份论文}

按照以下步骤创建第一份论文:

\begin{enumerate}
  \item 修改 \texttt{main.tex} 中的论文信息
  \item 编辑 \texttt{data/abstract.tex} 撰写中文摘要
  \item 编辑 \texttt{data/abstract\_en.tex} 撰写英文摘要
  \item 在 \texttt{data/} 目录创建章节文件
  \item 在 \texttt{ref/refs.bib} 添加参考文献
  \item 运行 \texttt{make thesis} 编译
\end{enumerate}

\chapter{高级功能}

\section{盲审模式}

生成盲审版本(隐藏作者和导师信息):

\begin{lstlisting}[language=TeX]
% 在 main.tex 第一行修改为:
\documentclass[master,review]{cumtbthesis}
\end{lstlisting}

盲审模式效果:
\begin{itemize}
  \item 封面作者姓名、学号留空
  \item 封面导师姓名、职称留空
  \item 不包含原创性声明和授权书
  \item 不包含致谢部分(可选)
\end{itemize}

\section{符号说明}

在 \texttt{main.tex} 的前置部分取消注释:

\begin{lstlisting}[language=TeX]
% 符号说明(可选)
\begin{notation}
$\alpha$ & 学习率参数 \\
$\beta$ & 衰减系数 \\
$\gamma$ & 折扣因子 \\
$\theta$ & 模型参数向量 \\
$\mathbf{X}$ & 输入数据矩阵 \\
$\mathbf{Y}$ & 输出数据矩阵 \\
$\mathbf{W}$ & 权重矩阵 \\
$\mathbf{b}$ & 偏置向量 \\
$\mathcal{L}$ & 损失函数 \\
$\nabla$ & 梯度算子 \\
$\mathbb{R}$ & 实数集 \\
$\mathbb{C}$ & 复数集 \\
\end{notation}

\end{lstlisting}

在 \texttt{data/notation.tex} 中添加符号:

\begin{lstlisting}[language=TeX]
\begin{notation}
$\alpha$ & 学习率参数 \\
$\beta$ & 衰减系数 \\
$\mathbf{X}$ & 输入数据矩阵 \\
\end{notation}
\end{lstlisting}

\section{图表使用}

\subsection{插入图片}

\begin{lstlisting}[language=TeX]
\begin{figure}[htbp]
  \centering
  \includegraphics[width=0.6\textwidth]{example.pdf}
  \caption{图片标题}
  \label{fig:example}
\end{figure}
\end{lstlisting}

引用图片:\verb|如图~\ref{fig:example} 所示|

\subsection{插入表格}

\begin{lstlisting}[language=TeX]
\begin{table}[htbp]
  \centering
  \caption{表格标题}
  \label{tab:example}
  \begin{tabular}{ccc}
    \toprule
    列1 & 列2 & 列3 \\
    \midrule
    数据1 & 数据2 & 数据3 \\
    \bottomrule
  \end{tabular}
\end{table}
\end{lstlisting}

\section{参考文献管理}

在 \texttt{ref/refs.bib} 中添加文献条目:

\begin{lstlisting}[language=TeX]
@article{example2024,
  title   = {Example Article},
  author  = {Zhang, San and Li, Si},
  journal = {Journal Name},
  volume  = {10},
  pages   = {123--145},
  year    = {2024}
}
\end{lstlisting}

在正文中引用:\verb|\cite{example2024}|

\chapter{常见问题}

\section{编译相关}

\subsection{Q: 编译失败,提示找不到文件?}

A: 确保:
\begin{enumerate}
  \item 文件路径正确
  \item 使用 UTF-8 编码保存文件
  \item 安装了所有必需的宏包
\end{enumerate}

\subsection{Q: 如何查看详细的错误信息?}

A: 查看 \texttt{main.log} 文件。

\subsection{Q: L3 kernel 版本不匹配警告?}

A: 这个警告不影响PDF生成,可以忽略。如要消除,运行:

\begin{lstlisting}[language=bash]
sudo fmtutil-sys --all
\end{lstlisting}

\section{格式相关}

\subsection{Q: 如何修改页边距?}

A: 在 \texttt{cumtbthesis.cls} 的 \verb|\geometry{}| 命令中修改。

\subsection{Q: 如何自定义章节标题样式?}

A: 修改 \texttt{cumtbthesis.cls} 中的 \verb|\ctexset{}| 配置。

\section{其他问题}

\subsection{Q: 如何贡献代码?}

A: 请参考 \texttt{CONTRIBUTING.md}(待创建)。

\subsection{Q: 如何反馈问题?}

A: 在 GitHub Issues 中提交问题报告。

\chapter{附录}

\section{Makefile 命令参考}

\begin{table}[htbp]
  \centering
  \caption{Makefile 命令列表}
  \begin{tabular}{ll}
    \toprule
    命令 & 说明 \\
    \midrule
    \texttt{make thesis} & 编译论文 PDF \\
    \texttt{make doc} & 编译使用手册 PDF \\
    \texttt{make all} & 编译所有文档 \\
    \texttt{make clean} & 清理辅助文件 \\
    \texttt{make cleanall} & 清理所有生成文件 \\
    \texttt{make distclean} & 完全清理 \\
    \texttt{make wordcount} & 统计字数 \\
    \texttt{make help} & 显示帮助信息 \\
    \bottomrule
  \end{tabular}
\end{table}

\section{相关资源}

\begin{itemize}
  \item 项目主页:\url{https://github.com/username/cumtb-thesis}
  \item \LaTeX{} 官网:\url{https://www.latex-project.org/}
  \item CTAN:\url{https://www.ctan.org/}
  \item CTeX 论坛:\url{http://bbs.ctex.org/}
\end{itemize}

\end{document}
