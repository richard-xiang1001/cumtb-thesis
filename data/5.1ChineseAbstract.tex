% --- 摘要标题部分 ---
\begingroup
    \vspace{\baselineskip} 
    \centering
    \zihao{-3}\heiti\bfseries 摘\quad\quad 要 
    \par
    \vspace{\baselineskip}
\endgroup

% --- 摘要正文部分 ---
\begingroup
    \zihao{-4}\songti
    \setlength{\baselineskip}{25pt} % 25磅行距
    \parindent 2em
    
    随着学术研究的日益精细化,传统的文字处理工具在处理复杂数学公式、自动化参考文献引用以及大规模文档一致性方面逐渐显现出局限性。本文以 \LaTeX{} 排版系统为核心,深入探讨了其在学术论文撰写中的效率优势与实现方法。文章首先简述了 \LaTeX{} 的基本逻辑及其与“所见即所得”模式的区别;随后结合中国矿业大学(北京)本科生毕业论文模版,详细阐述了文档结构拆分、字体字号控制、高质量图表制作以及数学公式输入的技巧。此外,本文还重点分析了利用 \textsf{BibTeX} 或索引列表自动化管理参考文献的方法,以确保引用格式的准确性。最后,通过本模版的实际应用案例,展示了如何通过简单的代码控制实现专业、美观、符合标准的学术论文排版。研究表明,熟练应用 \LaTeX{} 不仅能提升写作效率,更能确保学术成果呈现的高度严谨性。
    \par
\endgroup

% --- 关键词部分 ---
\vspace{2\baselineskip}

\begingroup
    \noindent 
    \zihao{-4}{\heiti\bfseries 关键词:}\songti \LaTeX{};学术写作;毕业论文;自动化排版;参考文献管理
\endgroup
\clearpage