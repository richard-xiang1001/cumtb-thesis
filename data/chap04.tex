% 子图使用示例

本章演示如何使用子图环境。

\section{使用 subcaption 宏包}

使用 \texttt{subcaption} 宏包可以创建并排的子图。

\subsection{基本用法}

\begin{figure}[htbp]
  \centering
  \begin{subfigure}[b]{0.45\textwidth}
    \centering
    % \includegraphics[width=\textwidth]{subfig1.pdf}
    \fbox{\parbox{0.9\textwidth}{\centering 子图 1\\(此处应为图片)}}
    \caption{第一幅子图}
    \label{fig:sub1}
  \end{subfigure}
  \hfill
  \begin{subfigure}[b]{0.45\textwidth}
    \centering
    % \includegraphics[width=\textwidth]{subfig2.pdf}
    \fbox{\parbox{0.9\textwidth}{\centering 子图 2\\(此处应为图片)}}
    \caption{第二幅子图}
    \label{fig:sub2}
  \end{subfigure}
  \caption{并排的两幅子图示例}
  \label{fig:subfigures}
\end{figure}

引用整个图组:图~\ref{fig:subfigures}。引用单个子图:图~\ref{fig:sub1} 和图~\ref{fig:sub2}。

\subsection{四宫格布局}

\begin{figure}[htbp]
  \centering
  \begin{subfigure}[b]{0.45\textwidth}
    \centering
    \fbox{\parbox{0.9\textwidth}{\centering 子图 A}}
    \caption{子图 A}
  \end{subfigure}
  \hfill
  \begin{subfigure}[b]{0.45\textwidth}
    \centering
    \fbox{\parbox{0.9\textwidth}{\centering 子图 B}}
    \caption{子图 B}
  \end{subfigure}
  
  \vspace{0.5cm}
  
  \begin{subfigure}[b]{0.45\textwidth}
    \centering
    \fbox{\parbox{0.9\textwidth}{\centering 子图 C}}
    \caption{子图 C}
  \end{subfigure}
  \hfill
  \begin{subfigure}[b]{0.45\textwidth}
    \centering
    \fbox{\parbox{0.9\textwidth}{\centering 子图 D}}
    \caption{子图 D}
  \end{subfigure}
  \caption{四宫格子图布局}
  \label{fig:grid}
\end{figure}

\subsection{自定义子图标签}

可以自定义子图的标签格式:

\begin{figure}[htbp]
  \centering
  \begin{subfigure}[b]{0.3\textwidth}
    \centering
    \fbox{\parbox{0.85\textwidth}{\centering 原图}}
    \caption{原图}
  \end{subfigure}
  \hfill
  \begin{subfigure}[b]{0.3\textwidth}
    \centering
    \fbox{\parbox{0.85\textwidth}{\centering 处理后}}
    \caption{处理结果}
  \end{subfigure}
  \hfill
  \begin{subfigure}[b]{0.3\textwidth}
    \centering
    \fbox{\parbox{0.85\textwidth}{\centering 对比}}
    \caption{对比图}
  \end{subfigure}
  \caption{图像处理流程示例}
  \label{fig:processing}
\end{figure}

\section{表格注释}

表格也可以有注释:

\begin{table}[htbp]
  \centering
  \caption{实验结果对比}
  \label{tab:results}
  \begin{tabular}{lcc}
    \toprule
    方法 & 准确率 (\%) & 时间 (s) \\
    \midrule
    方法 A & 85.2 & 10.3 \\
    方法 B & 87.5 & 12.1 \\
    方法 C\textsuperscript{*} & \textbf{90.1} & 15.8 \\
    \bottomrule
  \end{tabular}
  \vspace{0.2cm}
  
  \small \textsuperscript{*}注:方法C使用了额外的数据增强
\end{table}
