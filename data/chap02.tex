\chapter{模板设计与实现}

\section{设计原则}

本模板的设计遵循以下原则:

\begin{enumerate}
  \item \textbf{规范性}:严格遵循中国矿业大学(北京)的论文格式要求
  \item \textbf{易用性}:提供简洁的用户接口,降低使用门槛
  \item \textbf{可扩展性}:采用模块化设计,便于功能扩展
  \item \textbf{兼容性}:兼容主流的 \LaTeX{} 发行版
\end{enumerate}

\section{技术架构}

\subsection{文档类设计}

模板基于 ctexbook 文档类,这是 CTeX 宏集提供的中文书籍文档类,具有良好的中文支持。文档类的主要结构如图~\ref{fig:architecture} 所示。

% 这里可以添加一个架构图
% \begin{figure}[htbp]
%   \centering
%   \includegraphics[width=0.8\textwidth]{architecture.pdf}
%   \caption{模板技术架构}
%   \label{fig:architecture}
% \end{figure}

\subsection{宏包选择}

模板使用了以下主要宏包:

\begin{table}[htbp]
  \centering
  \caption{主要宏包及其用途}
  \label{tab:packages}
  \begin{tabular}{ll}
    \toprule
    宏包名称 & 用途 \\
    \midrule
    amsmath & 数学公式 \\
    graphicx & 图片插入 \\
    booktabs & 三线表 \\
    hyperref & 超链接 \\
    biblatex & 参考文献管理 \\
    listings & 代码高亮 \\
    \bottomrule
  \end{tabular}
\end{table}

\section{关键技术实现}

\subsection{页面布局}

根据学校要求,论文的页面设置为:
\begin{itemize}
  \item 纸张大小:A4 (210mm $\times$ 297mm)
  \item 上下边距:2.54cm
  \item 左右边距:3.17cm
  \item 页眉高度:1.5cm
  \item 页脚距离:1cm
\end{itemize}

使用 geometry 宏包可以方便地设置这些参数。

\subsection{数学公式}

模板支持各种数学公式,例如:

行内公式:$E = mc^2$

行间公式:
\begin{equation}
  \int_{-\infty}^{\infty} e^{-x^2} dx = \sqrt{\pi}
  \label{eq:gaussian}
\end{equation}

矩阵:
\begin{equation}
  \mathbf{A} = \begin{bmatrix}
    a_{11} & a_{12} & \cdots & a_{1n} \\
    a_{21} & a_{22} & \cdots & a_{2n} \\
    \vdots & \vdots & \ddots & \vdots \\
    a_{m1} & a_{m2} & \cdots & a_{mn}
  \end{bmatrix}
\end{equation}

\subsection{参考文献管理}

模板采用 biblatex + biber 进行参考文献管理,符合国家标准 GB/T 7714-2015。用户只需在 .bib 文件中添加文献条目,然后使用 \verb|\cite{}| 命令引用即可。
