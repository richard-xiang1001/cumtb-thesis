% ============================================================================
% 中国矿业大学(北京)本科生毕业设计(论文) LaTeX 模板 - 主文件
% ============================================================================
% [编译指引] 
% 推荐环境:TeX Live 2023+ 或 MiKTeX
% 编译方式:XeLaTeX (必须使用 XeLaTeX 才能正确处理中文字体)
% 文件编码:UTF-8
% ============================================================================

\documentclass[12pt, a4paper]{ctexart} % 使用 ctexart 文档类,预设了中文排版环境

\ExplSyntaxOn
\cs_if_exist:NF \tbl_save_outer_table_cols: { \cs_new:Npn \tbl_save_outer_table_cols: {} }
\ExplSyntaxOff

% ==================== 1. 宏包引入 (Packages) ====================
\usepackage{geometry}           % 设置页边距
\usepackage{fancyhdr}           % 自定义页眉页脚
\usepackage{titletoc}           % 目录样式定制核心包
\usepackage{fontspec}           % 字体设置支持
\usepackage{setspace}           % 控制行间距
\usepackage{amsmath}            % 数学公式支持
\usepackage{booktabs}           % 绘制三线表 (如 \toprule, \midrule)
\usepackage{caption}            % 定制图表标题样式
\usepackage{subcaption}         % 支持子图 (Subfigures)
\usepackage{cite}               % 参考文献引用处理
\usepackage{titlesec}           % 辅助设置标题格式
\usepackage{float}              % 固定图片位置 (使用 [H] 选项)
\usepackage{multirow}           % 表格中合并行
\usepackage{xcolor}             % 文本颜色支持
\usepackage{graphicx}           % 插入图片
\usepackage{tabularx}           % 自动调整列宽的表格
\usepackage{arydshln}           % 绘制虚线表格 (针对任务书等特殊页面)
\usepackage{tikz}               % 绘图工具 (用于绘制页面角标)
\usepackage{makecell}           % 单元格内换行支持
\usepackage{calc}               % 长度计算功能
\usepackage[hidelinks]{hyperref} % 超链接处理
\usepackage{enumitem}           % 列表环境定制 (用于参考文献等)

% 定义常用表格列格式
\newcolumntype{C}[1]{>{\centering\arraybackslash}p{#1}} % 居中且指定宽度
\newcolumntype{L}[1]{>{\raggedright\arraybackslash}p{#1}} % 左对齐且指定宽度

% ==================== 2. 全局页面设置 ====================
% 页面边距:符合标准版式要求
\geometry{left=3.18cm, right=3.18cm, top=2.54cm, bottom=2.54cm, headheight=15pt}

% 英文字体:默认使用 Times New Roman
\setmainfont{Times New Roman}

% 自定义字号命令 (bp为大点,pt为点)
\newcommand{\erhao}{\fontsize{22bp}{22bp}\selectfont}     % 二号
\newcommand{\sanhao}{\fontsize{16bp}{16bp}\selectfont}    % 三号
\newcommand{\xiaosan}{\fontsize{15pt}{22.5pt}\selectfont} % 小三
\newcommand{\xiaosi}{\fontsize{12pt}{18pt}\selectfont}    % 小四 (12pt)
\newcommand{\wuhao}{\fontsize{10.5pt}{12pt}\selectfont}   % 五号 (10.5pt)

% 行距定义
\newlength{\toclineskip}
\setlength{\toclineskip}{20pt} % 目录行间距
\newcommand{\bodylineskip}{25pt} % 正文行间距 (25磅)

% ==================== 3. 页眉页脚基本设置 ====================
\pagestyle{fancy}
\fancyhf{} 
\fancyhead[C]{\zihao{5}\songti 中国矿业大学(北京)20XX 届本科毕业设计(论文)}
\fancyfoot[C]{\zihao{5}\songti \thepage}
\renewcommand{\headrulewidth}{0.75pt}

% ==================== 3. 标题格式设置 (ctexset) ====================
% 定义各级标题的字体、字号和间距
\ctexset{
    section = {
        format = \centering\zihao{-3}\heiti\bfseries, % 一级标题:加粗、居中、小三、黑体
        aftername = \quad,
        beforeskip = 1\baselineskip,
        afterskip = 1\baselineskip,
        break = \clearpage, % 每一章从新页开始
    },
    subsection = {
        format = \raggedright\zihao{4}\heiti,         % 二级标题:四号、左对齐、黑体
        beforeskip = 0.5\baselineskip,
        afterskip = 0.5\baselineskip,
    },
    subsubsection = {
        format = \raggedright\zihao{-4}\songti\bfseries, % 三级标题:小四、左对齐、宋体加粗
        beforeskip = 0.5\baselineskip,
        afterskip = 0.5\baselineskip,
    }
}

% ==================== 4. 目录样式定义 (titletoc) ====================
\titlecontents{section}[0em]
    {\addvspace{0pt}\zihao{5}\songti\bfseries}
    {\thecontentslabel\hspace{1em}}
    {}
    {\titlerule*[0.5pc]{.}\contentspage}
    [\addvspace{0pt}]

\titlecontents{subsection}[2em]
    {\addvspace{0pt}\zihao{5}\songti}
    {\thecontentslabel\hspace{1em}}
    {}
    {\titlerule*[0.5pc]{.}\contentspage}
    [\addvspace{0pt}]

\titlecontents{subsubsection}[4em]
    {\addvspace{0pt}\zihao{5}\songti}
    {\thecontentslabel\hspace{1em}}
    {}
    {\titlerule*[0.5pc]{.}\contentspage}
    [\addvspace{0pt}]

% 重定义目录命令
\makeatletter
\renewcommand\tableofcontents{%
    \clearpage
    \thispagestyle{empty} 
    \begingroup
    \par
    \vspace{\baselineskip}
    \centering
    {\zihao{-3}\heiti\bfseries 目\quad\quad 录}
    \par
    \vspace{\baselineskip}
    \setlength{\baselineskip}{\toclineskip} 
    \@starttoc{toc}%
    \endgroup
    \clearpage
}
\makeatother

% ==================== 5. 文档开始 ====================
\begin{document}

% 封面
\pagestyle{empty} % 封面页不需要页码、页眉和页脚

% --- 1. 标题部分 (要求:二号、黑体、加粗、居中、单倍行距) ---
\begin{center}
    \setstretch{1.0} 
    \vspace*{2.0cm} % 顶部留空高度
    
    {\erhao\heiti\bfseries 中国矿业大学(北京)} \\
    \vspace{1.2cm}
    {\erhao\heiti\bfseries 本科生毕业设计(论文)}
\end{center}

\vspace{4.5cm} % 标题与表格之间的间距

% --- 2. 作者信息表格部分 (要求:三号、楷体、西文Times New Roman) ---
% 请在下方表格中填写您的个人信息
\begin{center}
    \setstretch{1.0}
    \sanhao\kaishu % 三号,楷体
    \setlength{\tabcolsep}{5pt}
    \renewcommand{\arraystretch}{2.1} % 调整行高以符合版式
    
    % 使用 tabularx 自动调整列宽,\hdashline 为虚线,\cline 为实线
    \begin{tabularx}{\textwidth}{;{1pt/1pt}l;{1pt/1pt}X;{1pt/1pt}l;{1pt/1pt}X;{1pt/1pt}}
        \hdashline
        中文题目: & \multicolumn{3}{l;{1pt/1pt}}{在此输入您的中文题目} \\ \cline{2-4}
        \hdashline
        英文题目: & \multicolumn{3}{l;{1pt/1pt}}{Enter your English title here} \\ \cline{2-4}
        \hdashline
        专题题目: & \multicolumn{3}{l;{1pt/1pt}}{在此输入专题题目(如有)} \\
        \hdashline
        姓\quad 名: & \centering 您的姓名 & 学\quad 号: & 您的学号 \\ \cline{2-2} \cline{4-4}
        \hdashline
        学\quad 院: & \multicolumn{3}{l;{1pt/1pt}}{在此输入学院名称} \\ \cline{2-4}
        \hdashline
        专\quad 业: & \centering 您的专业 & 班\quad 级: & 您的班级 \\ \cline{2-2} \cline{4-4}
        \hdashline
        指导教师: & \centering 教师姓名 & 职\quad 称: & 职称 \\ \cline{2-2} \cline{4-4}
        \hdashline
        完成日期: & \multicolumn{3}{c;{1pt/1pt}}{2025\hspace{1em} 年 \hspace{1em} 6 \hspace{1em} 月 \hspace{1em} 15 \hspace{1em} 日} \\ \cline{2-4}
        \hdashline
    \end{tabularx}
\end{center}
\clearpage

% 声明
\pagestyle{empty} % 不显示页码

% --- 第一部分:诚信声明 ---
\vspace*{0.5cm} 

\begin{center}
    % 标题:黑体,小三 (\zihao{-3}),取消标题缩进
    \hspace*{-\parindent}{\zihao{-3} \heiti 诚~信~声~明} 
\end{center}

\vspace{1cm}

\begin{spacing}{1.8} % 1.8倍行距
    % 正文:宋体,小四 (\zihao{-4})
    {\zihao{-4} \songti 本人郑重声明:所呈交的毕业设计(论文)是本人在指导教师的指导下独立完成的。除文中已经注明引用的内容外,毕业设计(论文)中不包含其他人已经发表或撰写过的研究成果。对本文做出重要贡献的个人和集体,均已在文中以明确方式标明。}
\end{spacing}

\vspace{0.8cm}

\begin{flushright}
    % 签名区通常不缩进,使用 flushright 靠右
    {\zihao{-4} \songti 作者签名:\underline{\hspace{3cm}} \quad 日期:\underline{\hspace{3cm}}}
\end{flushright}


\vspace{2.5cm} % 两个声明之间的垂直间距


% --- 第二部分:使用授权说明 ---

\begin{center}
    % 标题:黑体,小三,取消标题缩进
    \hspace*{-\parindent}{\zihao{-3} \heiti 关于使用授权的说明}
\end{center}

\vspace{1cm}

\begin{spacing}{1.8}
    % 正文:宋体,小四
    {\zihao{-4} \songti 本人完全了解中国矿业大学(北京)有关保留、使用毕业设计(论文)的规定,即:学校有权保留送交论文的复印件,允许论文被查阅或借阅;学校可以公布论文的全部或部分内容,可以采用影印、缩印或其他复制手段保存论文。}
\end{spacing}

\vspace{1.2cm}

% 签名档:取消缩进,并使用 \hfill 均匀分布
\noindent {\zihao{-4} \songti 作者签名:\underline{\hspace{2.2cm}} \hfill 导师签名:\underline{\hspace{2.2cm}} \hfill 日期:\underline{\hspace{2.2cm}}}
\clearpage

% 任务书
\pagestyle{empty} % 隐藏页码,确保页面干净(如需页码可删掉此行)
\enlargethispage{2cm} % 强制增加本页的可容纳高度


\begin{center}
    % 标题:黑体小三(包括括号)
    {\heiti\xiaosan 中国矿业大学 (北京) 本科生毕业设计 (论文) 任务书}
    
    \vspace{0.6em} % 稍微收缩标题下间距

    % 设置表格行高
    \setstretch{2.0} % 稍微减小一点行高确保不挤占空间

    % --- 第一部分:学院、专业、班级、姓名、学号 ---
    \begin{tabularx}{\textwidth}{:p{3.5em}:X:p{3em}:X:p{3em}:X:}
        \hdashline
        {\songti\xiaosi 学院:} & & {\songti\xiaosi 专业:} & & {\songti\xiaosi 班级:} & \\ \hdashline
        {\songti\xiaosi 姓名:} & & {\songti\xiaosi 学号:} & \multicolumn{3}{l:}{} \\ \hdashline
    \end{tabularx}
    
    \nointerlineskip

    % --- 第二部分:日期行(第一列宽度不同,垂直虚线错开) ---
    \begin{tabularx}{\textwidth}{:p{7.5em}:X:}
        {\songti\xiaosi 任务下达日期:} & 20 \hspace{1.5em} 年 \hspace{1.5em} 月 \hspace{1.5em} 日 \\ \hdashline
        {\songti\xiaosi 任务完成日期:} & 20 \hspace{1.5em} 年 \hspace{1.5em} 月 \hspace{1.5em} 日 \\ \hdashline
    \end{tabularx}

    \nointerlineskip

    % --- 第三部分:论文题目 ---
    \begin{tabularx}{\textwidth}{:p{7.5em}:X:}
        {\songti\xiaosi 论 文 题 目 :} & \\ \hdashline
        {\songti\xiaosi 专 题 题 目 :} & \\ \hdashline
    \end{tabularx}

    \nointerlineskip

    % --- 第四部分:任务主要内容 ---
    % 注意:这里的高度控制在 15.5cm 左右,通常可以刚好填满一页
    \begin{tabularx}{\textwidth}{:X:}
        \multicolumn{1}{:l:}{{\songti\xiaosi 任务主要内容:}} \\ \hdashline
        \begin{minipage}[t][15.5cm][t]{\dimexpr\textwidth-2\tabcolsep\relax}
            \vfill
        \end{minipage} \\ \hdashline
    \end{tabularx}
\end{center}
\clearpage
\pagestyle{empty} 
\enlargethispage{3cm} % 确保内容不溢出


\vspace*{-0.5cm} % 整体上移

\begin{center}
    \setstretch{1.6} 

    \begin{tabularx}{\textwidth}{:X:}
        \hdashline
        {\bfseries\songti\xiaosi 任务目标要求}{\songti\wuhao (文献阅读、外文资料翻译、设计或实验工作量,图纸、软硬件数量及技术指标等):} \\ \hdashline
        % 空白区域高度
        \begin{minipage}[t][7.5cm][t]{\dimexpr\textwidth-2\tabcolsep\relax}
            \hfill
        \end{minipage} \\ \hdashline
        
        {\bfseries\songti\xiaosi 时间进度安排:} \\ \hdashline
        \begin{minipage}[t][5.5cm][t]{\dimexpr\textwidth-2\tabcolsep\relax}
            \hfill
        \end{minipage} \\ \hdashline
        
        {\bfseries\songti\xiaosi 推荐阅读的文献资料:} \\ \hdashline
        \begin{minipage}[t][6.5cm][t]{\dimexpr\textwidth-2\tabcolsep\relax}
            \hfill
        \end{minipage} \\ \hdashline
    \end{tabularx}

    \nointerlineskip % 缝隙消除

    % 底部签字栏:左右边线也改为虚线 (:)
    \begin{tabularx}{\textwidth}{:p{8.5em}:X:p{8.5em}:X:}
        {\bfseries\songti\xiaosi 教学院长签字:} & & {\bfseries\songti\xiaosi 指导教师签字:} & \\ \hdashline
    \end{tabularx}
\end{center}
\clearpage

% 结合科研说明书
% \newgeometry{left=2.5cm, right=2.5cm, top=1.3cm, bottom=1.3cm} 
\pagestyle{empty}

% =============================================
% 2. 头部标题 (黑体小三)
% =============================================
\begin{center}
    \vspace*{-1.2cm} 
    {\zihao{-3}\heiti 中国矿业大学(北京)} \par
    \vspace{0.3em}
    {\zihao{-3}\heiti 本科生毕业设计(论文)结合科研说明书}
\end{center}

\vspace{-0.2em} % 缩减标题与“学院”的距离
\noindent {\zihao{-4}\songti 学院:} 
\vspace{-0.8em} % 显著缩减“学院”与表格的距离

\begin{center}
\zihao{-4}\songti 
\setlength{\tabcolsep}{0pt}
\renewcommand{\arraystretch}{1.3}

\noindent
\begin{tabular}{|C{2.5cm}|C{3.1cm}|C{2.4cm}|C{2.8cm}|C{2.4cm}|C{2.8cm}|}
\hline
学生姓名 &  & 专业 &  & 班级 &  \\ \hline
题目名称 & \multicolumn{5}{L{13.5cm}|}{} \\ \hline
题目种类 &  & 题目类型 & \multicolumn{3}{L{8.0cm}|}{} \\ \hline
指导教师 &  & 专业 &  & 职称 &  \\ \hline

\multirow{3}{*}{\parbox[c][5.8cm]{2.5cm}{\centering 科研课题 \\ 基本情况}} & 
\centering 科研课题名称 & \multicolumn{4}{L{10.4cm}|}{\rule{0pt}{0.85cm}} \\ \cline{2-6}
 & 
\centering 科研课题来源 & & \makecell[c]{科研立项\\起止时间} & \multicolumn{2}{C{5.2cm}|}{20\quad 年至 20\quad 年} \\ \cline{2-6}
 & 
\multicolumn{5}{L{13.5cm}|}{%
    \parbox[t][4.1cm][t]{13.5cm}{%
    \noindent\vbox{\kern2pt}主要研究内容:\\[3.6cm] 
    }} \\ \hline

\multirow{2}{*}{\parbox[c][4.6cm]{2.5cm}{\centering 学生参与 \\ 科研课题 \\ 研究情况}} & 
\multicolumn{5}{L{13.5cm}|}{%
    \parbox[t][2.8cm][t]{13.5cm}{%
    \noindent\vbox{\kern2pt}参与课题研究的内容:\\[2.3cm]
    }} \\ \cline{2-6}
 & 
\multicolumn{5}{L{13.5cm}|}{%
    \parbox[t][1.8cm][t]{13.5cm}{%
    \noindent\vbox{\kern2pt}参与研究的工作量:\\[1.3cm]
    }} \\ \hline

\multicolumn{3}{|c|}{系(教研室)意见} & \multicolumn{3}{c|}{学院意见} \\ \hline

\multicolumn{3}{|c|}{%
    \parbox[t][4.8cm][t]{7.8cm}{%
        \vspace{10pt}
        \hspace{2em}\zihao{5}毕业设计(论文)内容与科研课题相关,并且学生参与科研课题研究,同意认定毕业设计(论文)结合科研课题。
        \vfill
        \noindent \zihao{-4}系(教研室)主任签字:\\[1.5em] 
        \rightline{20\quad 年\quad 月\quad 日 \hspace{0.5em}}
        \vspace{8pt} % 留出底边安全间距
    }} & 
\multicolumn{3}{c|}{%
    \parbox[t][4.8cm][t]{7.8cm}{%
        \vspace{10pt}
        \hspace{2em}\zihao{5}毕业设计(论文)结合科研课题,并且符合学生创新学分认定要求,同意给予学生毕业设计(论文)结合科研创新学分。
        \vfill
        \noindent \zihao{-4}主管院长签字:\\[1.5em]
        \rightline{20\quad 年\quad 月\quad 日 \hspace{0.5em}}
        \vspace{8pt}
    }} \\ \hline
\end{tabular}
\end{center}

\restoregeometry 
\clearpage

\newgeometry{left=2.5cm, right=2.5cm, top=1.3cm, bottom=1.3cm} 
\pagestyle{empty}

% =============================================
% 2. 头部标题 (黑体小三)
% =============================================
\begin{center}
    \vspace*{-1.2cm} 
    {\zihao{-3}\heiti 中国矿业大学(北京)} \par
    \vspace{0.3em}
    {\zihao{-3}\heiti 本科生毕业设计(论文)结合科研说明书}
\end{center}

\vspace{-0.2em} % 缩减标题与“学院”的距离
\noindent {\zihao{-4}\songti 学院:} 
\vspace{-0.8em} % 显著缩减“学院”与表格的距离

\begin{center}
\zihao{-4}\songti 
\setlength{\tabcolsep}{0pt}
\renewcommand{\arraystretch}{1.3}

\noindent
\begin{tabular}{|C{2.5cm}|C{3.1cm}|C{2.4cm}|C{2.8cm}|C{2.4cm}|C{2.8cm}|}
\hline
学生姓名 &  & 专业 &  & 班级 &  \\ \hline
题目名称 & \multicolumn{5}{L{13.5cm}|}{} \\ \hline
题目种类 &  & 题目类型 & \multicolumn{3}{L{8.0cm}|}{} \\ \hline
指导教师 &  & 专业 &  & 职称 &  \\ \hline

\multirow{3}{*}{\parbox[c][5.8cm]{2.5cm}{\centering 科研课题 \\ 基本情况}} & 
\centering 科研课题名称 & \multicolumn{4}{L{10.4cm}|}{\rule{0pt}{0.85cm}} \\ \cline{2-6}
 & 
\centering 科研课题来源 & & \makecell[c]{科研立项\\起止时间} & \multicolumn{2}{C{5.2cm}|}{20\quad 年至 20\quad 年} \\ \cline{2-6}
 & 
\multicolumn{5}{L{13.5cm}|}{%
    \parbox[t][4.1cm][t]{13.5cm}{%
    \noindent\vbox{\kern2pt}主要研究内容:\\[3.6cm] 
    }} \\ \hline

\multirow{2}{*}{\parbox[c][4.6cm]{2.5cm}{\centering 学生参与 \\ 科研课题 \\ 研究情况}} & 
\multicolumn{5}{L{13.5cm}|}{%
    \parbox[t][2.8cm][t]{13.5cm}{%
    \noindent\vbox{\kern2pt}参与课题研究的内容:\\[2.3cm]
    }} \\ \cline{2-6}
 & 
\multicolumn{5}{L{13.5cm}|}{%
    \parbox[t][1.8cm][t]{13.5cm}{%
    \noindent\vbox{\kern2pt}参与研究的工作量:\\[1.3cm]
    }} \\ \hline

\multicolumn{3}{|c|}{系(教研室)意见} & \multicolumn{3}{c|}{学院意见} \\ \hline

\multicolumn{3}{|c|}{%
    \parbox[t][4.8cm][t]{7.8cm}{%
        \vspace{10pt}
        \hspace{2em}\zihao{5}毕业设计(论文)内容与科研课题相关,并且学生参与科研课题研究,同意认定毕业设计(论文)结合科研课题。
        \vfill
        \noindent \zihao{-4}系(教研室)主任签字:\\[1.5em] 
        \rightline{20\quad 年\quad 月\quad 日 \hspace{0.5em}}
        \vspace{8pt} % 留出底边安全间距
    }} & 
\multicolumn{3}{c|}{%
    \parbox[t][4.8cm][t]{7.8cm}{%
        \vspace{10pt}
        \hspace{2em}\zihao{5}毕业设计(论文)结合科研课题,并且符合学生创新学分认定要求,同意给予学生毕业设计(论文)结合科研创新学分。
        \vfill
        \noindent \zihao{-4}主管院长签字:\\[1.5em]
        \rightline{20\quad 年\quad 月\quad 日 \hspace{0.5em}}
        \vspace{8pt}
    }} \\ \hline
\end{tabular}
\end{center}

\restoregeometry 
\clearpage


% 中英文摘要
% --- 摘要标题部分 ---
\begingroup
    \vspace{\baselineskip} 
    \centering
    \zihao{-3}\heiti\bfseries 摘\quad\quad 要 
    \par
    \vspace{\baselineskip}
\endgroup

% --- 摘要正文部分 ---
\begingroup
    \zihao{-4}\songti
    \setlength{\baselineskip}{25pt} % 25磅行距
    \parindent 2em
    
    随着学术研究的日益精细化,传统的文字处理工具在处理复杂数学公式、自动化参考文献引用以及大规模文档一致性方面逐渐显现出局限性。本文以 \LaTeX{} 排版系统为核心,深入探讨了其在学术论文撰写中的效率优势与实现方法。文章首先简述了 \LaTeX{} 的基本逻辑及其与“所见即所得”模式的区别;随后结合中国矿业大学(北京)本科生毕业论文模版,详细阐述了文档结构拆分、字体字号控制、高质量图表制作以及数学公式输入的技巧。此外,本文还重点分析了利用 \textsf{BibTeX} 或索引列表自动化管理参考文献的方法,以确保引用格式的准确性。最后,通过本模版的实际应用案例,展示了如何通过简单的代码控制实现专业、美观、符合标准的学术论文排版。研究表明,熟练应用 \LaTeX{} 不仅能提升写作效率,更能确保学术成果呈现的高度严谨性。
    \par
\endgroup

% --- 关键词部分 ---
\vspace{2\baselineskip}

\begingroup
    \noindent 
    \zihao{-4}{\heiti\bfseries 关键词:}\songti \LaTeX{};学术写作;毕业论文;自动化排版;参考文献管理
\endgroup
\clearpage
% ==================== 标题部分 ====================
\begingroup
    \vspace{\baselineskip} 
    \centering
    \zihao{-3}\bfseries ABSTRACT
    \par
    \vspace{\baselineskip}
\endgroup

% ==================== 正文部分 ====================
\begingroup
    \zihao{-4}\rmfamily
    \setlength{\baselineskip}{25pt} % 25磅行距
    \parindent 2em
    
    With the increasing sophistication of academic research, traditional word processing tools show limitations in handling complex mathematical formulas, automated reference citations, and large-scale document consistency. This paper focuses on the \LaTeX{} typesetting system and deeply explores its efficiency advantages and implementation methods in academic writing. The article first outlines the basic logic of \LaTeX{} and its difference from the "What You See Is What You Get" mode. Then, combined with the undergraduate graduation thesis template of China University of Mining and Technology (Beijing), the techniques of document structure splitting, font size control, high-quality figure/table production, and mathematical formula input are explained in detail. In addition, this paper focuses on the methods of managing references automatically using \textsf{BibTeX} or index lists to ensure the accuracy of citation formats. Finally, through the practical application cases of this template, it demonstrates how to achieve professional, beautiful, and standard academic paper typesetting through simple code control. Research shows that proficiency in using \LaTeX{} not only improves writing efficiency but also ensures the rigorous presentation of academic achievements.
    \par
\endgroup

% ==================== 关键词部分 ====================
\vspace{2\baselineskip}

\begingroup
    \noindent 
    \zihao{-4}{\heiti\bfseries Keywords: }\rmfamily \LaTeX{}; Academic Writing; Bachelor Thesis; Automated Typesetting; Reference Management
\endgroup
\clearpage

% 目录
\tableofcontents

% 正文
\pagenumbering{arabic}
\pagestyle{fancy}
% 第1章 绪论
\section{绪论}
\begingroup
\zihao{-4}\songti
\setlength{\baselineskip}{25pt}
\parindent 2em

\subsection{研究背景与意义}
自 Leslie Lamport 在 20 世纪 80 年代开发以来,\LaTeX{} 已成为理工科学术交流事实上的标准\cite{ref-lamport}。在毕业论文撰写过程中,由于涉及大量的多级标题、交叉引用、自动化目录以及复杂的数学推导,传统的文字处理软件(如 Microsoft Word)往往因其排版逻辑的局限性,导致用户在后期格式调整上耗费巨大精力。

使用 \LaTeX{} 的核心意义在于“内容与格式的分离”。用户只需关注学术内容的表达,而将复杂的排版规则交给宏包与编译器处理。尤其是在中国矿业大学(北京)的毕业设计(论文)规范要求下,统一且严谨的格式是获得优秀评价的重要前提\cite{ref-knuth1}。

\subsection{自动化管理的优雅性}
\LaTeX{} 的自动化特性体现在其对文档要素的精准控制。无论是目录的实时更新、公式引用的自动编号,还是参考文献的格式化排版,都无需人工干预。这种“一次配置,全局生效”的逻辑,正是其魅力所在。

\endgroup

% 第2章 论文主体:LaTeX 核心技术实践
\clearpage
\section{论文主体}
\begingroup
\zihao{-4}\songti
\setlength{\baselineskip}{25pt}
\parindent 2em

\subsection{数学公式的深度解析}
\LaTeX{} 在数学式表达上的优雅是无与伦比的。除了基础的行内公式和简单方程,它还能轻松处理复杂的多行推导和矩阵运算。

\subsubsection{多行公式推导与对齐}
在物理或数学推演中,常常需要多行公式对齐。使用 \texttt{aligned} 环境可以让等号精准对齐,如式\ref{eq:multi-line}所示:
\begin{equation}
    \begin{aligned}
        \nabla \cdot \mathbf{E} &= \frac{\rho}{\varepsilon_0} \\
        \nabla \cdot \mathbf{B} &= 0 \\
        \nabla \times \mathbf{E} &= -\frac{\partial \mathbf{B}}{\partial t} \\
        \nabla \times \mathbf{B} &= \mu_0\left(\mathbf{J} + \varepsilon_0\frac{\partial \mathbf{E}}{\partial t}\right)
    \end{aligned}
    \label{eq:multi-line}
\end{equation}

\subsubsection{矩阵与复杂算子}
处理大规模矩阵时,\LaTeX{} 的语法也极其直观:
\begin{equation}
    \mathbf{A} = \begin{bmatrix}
        a_{11} & a_{12} & \cdots & a_{1n} \\
        a_{21} & a_{22} & \cdots & a_{2n} \\
        \vdots & \vdots & \ddots & \vdots \\
        a_{m1} & a_{m2} & \cdots & a_{mn}
    \end{bmatrix}
\end{equation}

\subsection{图表与浮动体的高级应用}
\LaTeX{} 能够智能安排图表位置,确保页面排版的紧凑与美观。

\subsubsection{子图并排显示}
在对比实验结果时,通常需要将几张图放在同一个标题下。通过 \texttt{minipage} 或相关的子图宏包,可以实现如图\ref{fig:comparison} 所示的布局。
\begin{figure}[H]
    \centering
    \begin{minipage}{0.45\textwidth}
        \centering
        \fbox{实验组 A 图片}
        \caption{实验组 A 结果}
    \end{minipage}
    \hfill
    \begin{minipage}{0.45\textwidth}
        \centering
        \fbox{实验组 B 图片}
        \caption{实验组 B 结果}
    \end{minipage}
    \caption{两组实验结果的对比分析}
    \label{fig:comparison}
\end{figure}

\subsubsection{复杂表格的灵活性}
除了基础的三线表,\LaTeX{} 还能处理合并行、合并列以及跨页的长表格。表\ref{tab:complex} 展示了一个包含合并单元格的表格,其结构远比手动绘制稳定。

\begin{table}[H]
    \centering
    \caption{煤样力学性能综合测试表}
    \label{tab:complex}
    \zihao{5}\songti
    \begin{tabular}{ccccc}
        \toprule
        \multirow{2}{*}{采样地点} & \multicolumn{2}{c}{物理性质} & \multicolumn{2}{c}{力学指标} \\
        \cmidrule(lr){2-3} \cmidrule(lr){4-5}
        & 密度 (g/cm$^3$) & 含水率 (\%) & 抗压强度 (MPa) & 弹性模量 (GPa) \\
        \midrule
        东区 1011 & 1.35 & 3.2 & 24.5 & 3.5 \\
        西区 2022 & 1.42 & 2.8 & 28.1 & 3.9 \\
        \bottomrule
    \end{tabular}
\end{table}

\subsection{参考文献:从手动到自动的跃迁}
正如 Knuth 在《数字书法》\cite{ref-knuth2}中所言,排版应当是学术诚信的延伸。手动维护几十个参考文献不仅枯燥,而且极易出错。

在本模版中,只需使用 \verb|\cite{label}| 命令,引用的顺序标号就会根据你在列表中的位置自动调整。例如,我们在补充了关于参考文献著录规则的国家标准\cite{ref-gb7714}后,后续的所有文献序号都会自动顺延,无需手动修改正文中的数字。

这种“一次一处修改,全局自动对齐”的逻辑,是 \LaTeX{} 能够成为学术研究利器的根本原因。

\endgroup

% --- 致谢 ---
\clearpage
\phantomsection
\addcontentsline{toc}{section}{致谢}
\begin{center}
    {\zihao{-3}\heiti\bfseries 致\quad 谢}
\end{center}
\vspace{\baselineskip}
\begingroup
\zihao{-4}\songti
\setlength{\baselineskip}{25pt}
\hspace*{2em}在完成本篇 \LaTeX{} 使用教程的过程中,感谢 Leslie Lamport 和 Donald Knuth 为学术界带来的伟大工具。同时也感谢中国矿业大学(北京)对本科生学术规范的支持,使我们能在一个科学、严谨的环境下完成学业。
\endgroup

% --- 参考文献 ---
\clearpage
\phantomsection
\addcontentsline{toc}{section}{参考文献}
\begingroup
    \noindent
    {\zihao{-3}\heiti\bfseries 参考文献} \par
\endgroup
\vspace{-1.0\baselineskip}
\begingroup
    \zihao{5}\songti
    \setlength{\baselineskip}{20pt}
    \renewcommand{\refname}{} % 禁用环境默认标题
    \begin{thebibliography}{99}
        \addtolength{\itemsep}{-0.5em} % 紧凑排版
        \bibitem{ref-lamport} LAMPORT L. \LaTeX: A Document Preparation System[M]. 2nd ed. Reading: Addison-Wesley, 1994.
        \bibitem{ref-knuth1} KNUTH D E. The \TeX{}book[M]. Reading: Addison-Wesley Professional, 1984.
        \bibitem{ref-knuth2} KNUTH D E. Digital Typography[M]. Stanford: Center for the Study of Language and Information, 1999.
        \bibitem{ref-kopka} KOPKA H, DALY P W. Guide to \LaTeX[M]. 4th ed. Boston: Addison-Wesley, 2003.
        \bibitem{ref-gb7714} 中华人民共和国国家质量监督检验检疫总局, 中国国家标准化管理委员会. 信息与文献:参考文献著录规则: GB/T 7714—2015[S]. 北京: 中国标准出版社, 2015.
        \bibitem{ref-liu} 刘国钧, 陈绍业, 王凤翥, 等. 图书馆目录[M]. 北京: 高等教育出版社, 1957.
    \end{thebibliography}
\endgroup

% --- 附录 ---
\clearpage
\phantomsection
\addcontentsline{toc}{section}{附录 1\hspace{1em}常用 LaTeX 宏包清单}
\noindent {\zihao{-3}\heiti\bfseries 附录 1\hspace{1em}常用 LaTeX 宏包清单}
\vspace{\baselineskip}

\begingroup
\zihao{-4}\songti
\setlength{\baselineskip}{25pt}
本模版已内置并测试通过了以下关键宏包:
\begin{enumerate}
    \item \texttt{amsmath}: 数学公式核心增强宏包。
    \item \texttt{booktabs}: 实现学术三线表。
    \item \texttt{multirow}: 处理表格合并行。
    \item \texttt{graphicx}: 插入图像文件的标准接口。
    \item \texttt{geometry}: 调整页边距与页面布局。
\end{enumerate}

\textcolor{red}{【附录内容如包含代码,建议使用 \texttt{listings} 环境以获得更好的代码高亮支持。】}
\endgroup

\end{document}
