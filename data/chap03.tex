\chapter{模板使用说明}

\section{环境配置}

\subsection{安装 \LaTeX{} 发行版}

使用本模板需要先安装 \LaTeX{} 发行版,推荐:

\begin{itemize}
  \item \textbf{Windows}:TeX Live 或 MiKTeX
  \item \textbf{macOS}:MacTeX (基于 TeX Live)
  \item \textbf{Linux}:TeX Live
\end{itemize}

建议安装完整版(Full Scheme),确保包含所有需要的宏包。

\subsection{编辑器选择}

可以使用以下编辑器编写 \LaTeX{} 文档:

\begin{itemize}
  \item TeXworks(TeX Live 自带)
  \item TeXstudio
  \item VS Code + LaTeX Workshop 插件
  \item Overleaf(在线编辑器)
\end{itemize}

\section{编译方法}

本模板使用 XeLaTeX + Biber 编译,完整的编译流程为:

\begin{verbatim}
xelatex main.tex
biber main
xelatex main.tex
xelatex main.tex
\end{verbatim}

\textbf{说明:}
\begin{enumerate}
  \item 第一次 xelatex 编译生成 .aux 和 .bcf 文件
  \item biber 处理参考文献
  \item 第二次 xelatex 更新参考文献引用
  \item 第三次 xelatex 更新目录和交叉引用
\end{enumerate}

\section{基本使用}

\subsection{配置论文信息}

在 main.tex 中配置论文的基本信息:

\begin{lstlisting}[language=TeX]
\ctitle{论文题目}
\cauthor{作者姓名}
\studentid{学号}
\school{学院名称}
\major{专业名称}
\advisor{导师姓名}
\advisortitle{导师职称}
\cdate{2025年6月}

\keywords{关键词1;关键词2;关键词3}
\enkeywords{keyword1; keyword2; keyword3}
\end{lstlisting}

\subsection{撰写章节内容}

在 data/ 目录下创建章节文件,如 chap01.tex、chap02.tex 等,然后在 main.tex 中使用 \verb|\input{}| 命令引入。

\subsection{插入图片}

图片文件放在 figures/ 目录下,使用以下代码插入:

\begin{lstlisting}[language=TeX]
\begin{figure}[htbp]
  \centering
  \includegraphics[width=0.6\textwidth]{example.pdf}
  \caption{图片标题}
  \label{fig:example}
\end{figure}
\end{lstlisting}

引用图片:\verb|如图~\ref{fig:example} 所示|

\subsection{添加参考文献}

在 ref/refs.bib 中添加文献条目,然后在正文中使用 \verb|\cite{}| 命令引用。

\section{高级功能}

\subsection{算法环境}

模板支持算法环境,可以编写伪代码:

\begin{algorithm}[htbp]
  \caption{示例算法}
  \label{alg:example}
  \begin{algorithmic}[1]
    \REQUIRE 输入数据 $x$
    \ENSURE 输出结果 $y$
    \STATE 初始化 $y \leftarrow 0$
    \FOR{$i = 1$ to $n$}
      \STATE $y \leftarrow y + x_i$
    \ENDFOR
    \RETURN $y$
  \end{algorithmic}
\end{algorithm}

\subsection{代码高亮}

使用 listings 宏包可以插入带语法高亮的代码:

\begin{lstlisting}[language=Python, caption=Python代码示例]
def factorial(n):
    if n <= 1:
        return 1
    else:
        return n * factorial(n - 1)

result = factorial(5)
print(f"5! = {result}")
\end{lstlisting}

\section{常见问题}

\subsection{编译错误}

如果遇到编译错误,可以尝试:
\begin{enumerate}
  \item 检查 \LaTeX{} 发行版是否完整安装
  \item 确认使用 XeLaTeX 编译引擎
  \item 检查文献库文件的编码是否为 UTF-8
  \item 删除辅助文件后重新编译
\end{enumerate}

\subsection{字体问题}

如果遇到字体问题,确认系统已安装宋体、黑体、楷体等中文字体,以及 Times New Roman 等英文字体。
