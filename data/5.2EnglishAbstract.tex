% ==================== 标题部分 ====================
\begingroup
    \vspace{\baselineskip} 
    \centering
    \zihao{-3}\bfseries ABSTRACT
    \par
    \vspace{\baselineskip}
\endgroup

% ==================== 正文部分 ====================
\begingroup
    \zihao{-4}\rmfamily
    \setlength{\baselineskip}{25pt} % 25磅行距
    \parindent 2em
    
    With the increasing sophistication of academic research, traditional word processing tools show limitations in handling complex mathematical formulas, automated reference citations, and large-scale document consistency. This paper focuses on the \LaTeX{} typesetting system and deeply explores its efficiency advantages and implementation methods in academic writing. The article first outlines the basic logic of \LaTeX{} and its difference from the "What You See Is What You Get" mode. Then, combined with the undergraduate graduation thesis template of China University of Mining and Technology (Beijing), the techniques of document structure splitting, font size control, high-quality figure/table production, and mathematical formula input are explained in detail. In addition, this paper focuses on the methods of managing references automatically using \textsf{BibTeX} or index lists to ensure the accuracy of citation formats. Finally, through the practical application cases of this template, it demonstrates how to achieve professional, beautiful, and standard academic paper typesetting through simple code control. Research shows that proficiency in using \LaTeX{} not only improves writing efficiency but also ensures the rigorous presentation of academic achievements.
    \par
\endgroup

% ==================== 关键词部分 ====================
\vspace{2\baselineskip}

\begingroup
    \noindent 
    \zihao{-4}{\heiti\bfseries Keywords: }\rmfamily \LaTeX{}; Academic Writing; Bachelor Thesis; Automated Typesetting; Reference Management
\endgroup
\clearpage