% 第1章 绪论
\section{绪论}
\begingroup
\zihao{-4}\songti
\setlength{\baselineskip}{25pt}
\parindent 2em

\subsection{研究背景与意义}
自 Leslie Lamport 在 20 世纪 80 年代开发以来,\LaTeX{} 已成为理工科学术交流事实上的标准\cite{ref-lamport}。在毕业论文撰写过程中,由于涉及大量的多级标题、交叉引用、自动化目录以及复杂的数学推导,传统的文字处理软件(如 Microsoft Word)往往因其排版逻辑的局限性,导致用户在后期格式调整上耗费巨大精力。

使用 \LaTeX{} 的核心意义在于“内容与格式的分离”。用户只需关注学术内容的表达,而将复杂的排版规则交给宏包与编译器处理。尤其是在中国矿业大学(北京)的毕业设计(论文)规范要求下,统一且严谨的格式是获得优秀评价的重要前提\cite{ref-knuth1}。

\subsection{自动化管理的优雅性}
\LaTeX{} 的自动化特性体现在其对文档要素的精准控制。无论是目录的实时更新、公式引用的自动编号,还是参考文献的格式化排版,都无需人工干预。这种“一次配置,全局生效”的逻辑,正是其魅力所在。

\endgroup

% 第2章 论文主体:LaTeX 核心技术实践
\clearpage
\section{论文主体}
\begingroup
\zihao{-4}\songti
\setlength{\baselineskip}{25pt}
\parindent 2em

\subsection{数学公式的深度解析}
\LaTeX{} 在数学式表达上的优雅是无与伦比的。除了基础的行内公式和简单方程,它还能轻松处理复杂的多行推导和矩阵运算。

\subsubsection{多行公式推导与对齐}
在物理或数学推演中,常常需要多行公式对齐。使用 \texttt{aligned} 环境可以让等号精准对齐,如式\ref{eq:multi-line}所示:
\begin{equation}
    \begin{aligned}
        \nabla \cdot \mathbf{E} &= \frac{\rho}{\varepsilon_0} \\
        \nabla \cdot \mathbf{B} &= 0 \\
        \nabla \times \mathbf{E} &= -\frac{\partial \mathbf{B}}{\partial t} \\
        \nabla \times \mathbf{B} &= \mu_0\left(\mathbf{J} + \varepsilon_0\frac{\partial \mathbf{E}}{\partial t}\right)
    \end{aligned}
    \label{eq:multi-line}
\end{equation}

\subsubsection{矩阵与复杂算子}
处理大规模矩阵时,\LaTeX{} 的语法也极其直观:
\begin{equation}
    \mathbf{A} = \begin{bmatrix}
        a_{11} & a_{12} & \cdots & a_{1n} \\
        a_{21} & a_{22} & \cdots & a_{2n} \\
        \vdots & \vdots & \ddots & \vdots \\
        a_{m1} & a_{m2} & \cdots & a_{mn}
    \end{bmatrix}
\end{equation}

\subsection{图表与浮动体的高级应用}
\LaTeX{} 能够智能安排图表位置,确保页面排版的紧凑与美观。

\subsubsection{子图并排显示}
在对比实验结果时,通常需要将几张图放在同一个标题下。通过 \texttt{minipage} 或相关的子图宏包,可以实现如图\ref{fig:comparison} 所示的布局。
\begin{figure}[H]
    \centering
    \begin{minipage}{0.45\textwidth}
        \centering
        \fbox{实验组 A 图片}
        \caption{实验组 A 结果}
    \end{minipage}
    \hfill
    \begin{minipage}{0.45\textwidth}
        \centering
        \fbox{实验组 B 图片}
        \caption{实验组 B 结果}
    \end{minipage}
    \caption{两组实验结果的对比分析}
    \label{fig:comparison}
\end{figure}

\subsubsection{复杂表格的灵活性}
除了基础的三线表,\LaTeX{} 还能处理合并行、合并列以及跨页的长表格。表\ref{tab:complex} 展示了一个包含合并单元格的表格,其结构远比手动绘制稳定。

\begin{table}[H]
    \centering
    \caption{煤样力学性能综合测试表}
    \label{tab:complex}
    \zihao{5}\songti
    \begin{tabular}{ccccc}
        \toprule
        \multirow{2}{*}{采样地点} & \multicolumn{2}{c}{物理性质} & \multicolumn{2}{c}{力学指标} \\
        \cmidrule(lr){2-3} \cmidrule(lr){4-5}
        & 密度 (g/cm$^3$) & 含水率 (\%) & 抗压强度 (MPa) & 弹性模量 (GPa) \\
        \midrule
        东区 1011 & 1.35 & 3.2 & 24.5 & 3.5 \\
        西区 2022 & 1.42 & 2.8 & 28.1 & 3.9 \\
        \bottomrule
    \end{tabular}
\end{table}

\subsection{参考文献:从手动到自动的跃迁}
正如 Knuth 在《数字书法》\cite{ref-knuth2}中所言,排版应当是学术诚信的延伸。手动维护几十个参考文献不仅枯燥,而且极易出错。

在本模版中,只需使用 \verb|\cite{label}| 命令,引用的顺序标号就会根据你在列表中的位置自动调整。例如,我们在补充了关于参考文献著录规则的国家标准\cite{ref-gb7714}后,后续的所有文献序号都会自动顺延,无需手动修改正文中的数字。

这种“一次一处修改,全局自动对齐”的逻辑,是 \LaTeX{} 能够成为学术研究利器的根本原因。

\endgroup

% --- 致谢 ---
\clearpage
\phantomsection
\addcontentsline{toc}{section}{致谢}
\begin{center}
    {\zihao{-3}\heiti\bfseries 致\quad 谢}
\end{center}
\vspace{\baselineskip}
\begingroup
\zihao{-4}\songti
\setlength{\baselineskip}{25pt}
\hspace*{2em}在完成本篇 \LaTeX{} 使用教程的过程中,感谢 Leslie Lamport 和 Donald Knuth 为学术界带来的伟大工具。同时也感谢中国矿业大学(北京)对本科生学术规范的支持,使我们能在一个科学、严谨的环境下完成学业。
\endgroup

% --- 参考文献 ---
\clearpage
\phantomsection
\addcontentsline{toc}{section}{参考文献}
\begingroup
    \noindent
    {\zihao{-3}\heiti\bfseries 参考文献} \par
\endgroup
\vspace{-1.0\baselineskip}
\begingroup
    \zihao{5}\songti
    \setlength{\baselineskip}{20pt}
    \renewcommand{\refname}{} % 禁用环境默认标题
    \begin{thebibliography}{99}
        \addtolength{\itemsep}{-0.5em} % 紧凑排版
        \bibitem{ref-lamport} LAMPORT L. \LaTeX: A Document Preparation System[M]. 2nd ed. Reading: Addison-Wesley, 1994.
        \bibitem{ref-knuth1} KNUTH D E. The \TeX{}book[M]. Reading: Addison-Wesley Professional, 1984.
        \bibitem{ref-knuth2} KNUTH D E. Digital Typography[M]. Stanford: Center for the Study of Language and Information, 1999.
        \bibitem{ref-kopka} KOPKA H, DALY P W. Guide to \LaTeX[M]. 4th ed. Boston: Addison-Wesley, 2003.
        \bibitem{ref-gb7714} 中华人民共和国国家质量监督检验检疫总局, 中国国家标准化管理委员会. 信息与文献:参考文献著录规则: GB/T 7714—2015[S]. 北京: 中国标准出版社, 2015.
        \bibitem{ref-liu} 刘国钧, 陈绍业, 王凤翥, 等. 图书馆目录[M]. 北京: 高等教育出版社, 1957.
    \end{thebibliography}
\endgroup

% --- 附录 ---
\clearpage
\phantomsection
\addcontentsline{toc}{section}{附录 1\hspace{1em}常用 LaTeX 宏包清单}
\noindent {\zihao{-3}\heiti\bfseries 附录 1\hspace{1em}常用 LaTeX 宏包清单}
\vspace{\baselineskip}

\begingroup
\zihao{-4}\songti
\setlength{\baselineskip}{25pt}
本模版已内置并测试通过了以下关键宏包:
\begin{enumerate}
    \item \texttt{amsmath}: 数学公式核心增强宏包。
    \item \texttt{booktabs}: 实现学术三线表。
    \item \texttt{multirow}: 处理表格合并行。
    \item \texttt{graphicx}: 插入图像文件的标准接口。
    \item \texttt{geometry}: 调整页边距与页面布局。
\end{enumerate}

\textcolor{red}{【附录内容如包含代码,建议使用 \texttt{listings} 环境以获得更好的代码高亮支持。】}
\endgroup